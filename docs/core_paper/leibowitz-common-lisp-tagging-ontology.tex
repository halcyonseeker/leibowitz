\documentclass[12pt, twoside]{article}

\usepackage[utf8]{inputenc}
\usepackage[T1]{fontenc}
\usepackage{hyperref}
\usepackage[margin=0.5in]{geometry}

\usepackage[backend=biber,style=mla,citestyle=authoryear,sorting=none]{biblatex}
\addbibresource{references.bib}

\begin{document}

\title{Leibowitz: A Tagging Ontology In Common Lisp}
\author{Thalia Wright}
\maketitle

\hypersetup{
  pdfauthor={Thalia Wright},
  pdftitle={Leibowitz: A Tagging Ontology In Common Lisp},
  pdfsubject={Technical overview of a proof-of-concept Common Lisp library},
  pdfcreator={LaTeX via AUCTeX in GNU Emacs},
  pdflang={English}}

\abstract{The Leibowitz core is a modular Common Lisp system that
  implements an ontology for organizing units of data into
  mutually-inclusive hierarchies of tags.  Its reference client
  program is a file tagging and full-text search utility so by default
  it provides backends that source data from files on a file system
  and queries an ontology stored in a SQLite database.  It's designed
  in a modular fashion to allow programmers to use it to add rich
  tagging to any kind of dataset.}

%%%%%%%%%%%%%%%%%%%%%%%%%%%%%%%%%%%%%%%%%%%%%%%%%%%%%%%%%%%%%%%%%%%%%%%%
\section{Rational and Background}

Leibowitz grew out of a frustration with the limitations of the
organizational utility of Unix-style hierarchical file systems.

%%%%%%%%%%%%%%%%%%%%%%%%%%%%%%%%%%%%%%%%%%%%%%%%%%%%%%%%%%%%%%%%%%%%%%%%
\section{Overview of Tagging Systems}

%%%%%%%%%%%%%%%%%%%%%%%%%%%%%%%%%%%%%%%%%%%%%%%%%%%%%%%%%%%%%%%%%%%%%%%%
\section{Why Common Lisp?}

%%%%%%%%%%%%%%%%%%%%%%%%%%%%%%%%%%%%%%%%%%%%%%%%%%%%%%%%%%%%%%%%%%%%%%%%
\section{API Overview}

%%%%%%%%%%%%%%%%%%%%%%%%%%%%%%%%%%%%%%%%%%%%%%%%%%%%%%%%%%%%%%%%%%%%%%%%
\section{Code Structure}

%%%%%%%%%%%%%%%%%%%%%%%%%%%%%%%%%%%%%%%%%%%%%%%%%%%%%%%%%%%%%%%%%%%%%%%%
\section{Future Directions}

\subsection{Building Folksonomies}

\subsection{Building Expert Systems}

%%%%%%%%%%%%%%%%%%%%%%%%%%%%%%%%%%%%%%%%%%%%%%%%%%%%%%%%%%%%%%%%%%%%%%%%
\section{Concrete Applications}

%%%%%%%%%%%%%%%%%%%%%%%%%%%%%%%%%%%%%%%%%%%%%%%%%%%%%%%%%%%%%%%%%%%%%%%%
\section{Citations}

\printbibliography

\end{document}
